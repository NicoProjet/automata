\section{Introduction}
Les automates sont souvent décrits comme des machines donnant une réponse suivant une entrée à fournir. De manière plus précise et scientifique, ce sont des 5-tuples 
\[ M = (Q,\Sigma ,q_{0},\delta ,F) \]
avec respectivement un ensemble d'états, un alphabet des symboles d'entrée, un état initial, une fonction de transition et un ensemble d'états acceptants. On commence toujours par l'état initial, et on voyage dans les états de $\Sigma$ en suivant la fonction $\delta$ pour les valeurs en entrée.\par
Cependant cet article se concentre sur le test du vide des automates à compteur avec limite d'inversion. Ce sont des automates finis \footnote{un nombre fini d'états} déterministe \footnote{pour un état et une entrée, la fonction $\delta$ donnera toujours le même résultat} à compteurs \footnote{des compteurs sont maintenus et incrémenté/décrémenté lors de changement d'états, ils peuvent être comparés à 0 dans la fonction $\delta$} avec limite d'inversion \footnote{les compteurs peuvent changer de sens incrémentation <--> décrémentation un nombre fini de fois}, ils peuvent être décrits par un 6-tuple 
\[ M = (k, Q, \Sigma, \delta, q_{0}, F) \]
respectivement la limite sur les compteurs, les états, les entrées, la fonction de transition, l'état initial et les états acceptants. Ces contraintes simplifient le problème étudié (test du vide) et notamment la limite d'inversion qui le rend décidable en temps polynomiale pour les automates à compteur en ajoutant une limite à la {\bf taille} des mots d'entrée (détaillé dans un article d'Oscar H. Ibarra, Journal 22). 