\begin{abstract}
    Les automates à compteur avec limite d'inversion sont déjà connus et introduits dans un article de Oscar H. Ibarra. Le but de ce mini-mémoire était de chercher et implémenter un test du vide pour ces derniers. Est-ce un test réalisable? En quelle complexité? Quels éléments rendent le problème décidable? Ce sont toutes des questions auxquelles on va tenter de répondre lors de ce travail. \par
    L'article nous emmène donc dans la réflexion et l'implémentation des différentes stratégies imaginées et mises en oeuvre.\par
    Pour résumer le résultat en quelques mots, nous avons un algorithme assez simple et très efficace pour de petits automates, lorsque cet automate se complique, le test du vide peut très vite devenir complexe et long. Il est bien sûr possible d'ajouter plus d'améliorations et il ne faut pas oublier que chaque algorithme peut être plus efficace que l'autre suivant l'automate en entrée.
    
    
\end{abstract}